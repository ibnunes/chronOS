%%%%%%%%%%%%%%%%%%%%%%%%%%%%%%%%%%%%%%%%%%%%%%%%%%%%%%%%%%%%%%%%%%%%%%%%%%%%%%%%%%%%%%%%%%%%%%%%%%%%%%%%%%%%%%
%% Este é o ficheiro formatacaoUBI.tex - NÃO EDITAR excepto a secção hypersetup!
%% Define a formatação a ser usada em teses apresentadadas na Universidade da Beira Interior, seguindo o despacho Reitoral nº 49/R/2010
%% Versão 2.2 - 01/06/2016 - Podem aparecer as palavras "Figura" e "Tabela" nas respectivas listas
%% Versão 2.1 - 28/03/2014 - Agora compila com o XeLaTeX por causa do tipo de fonte, incluido o estilo de biblipgrafia IEEE, possibilidade de escolha de tipo de fonte matemático
%% Versão 2.0 - 10/11/2011 - Bibliografia agora aparece no índice
%% Versão 1.9 - 10/10/2011 - Resolvido problema em que o texto nas tabelas aparecia em cima da linha superior
%% Versão 1.8 - 12/07/2011 - Legendas são agora centradas
%% Versão 1.7 - 8/07/2011 - Correcção de algumas medidas de acordo com novo modelo de Word
%% Versão 1.6 - 1/07/2011 - Trebuchet inserido como fonte principal
%% Adaptado do original de Oliver Commowick para estar de acordo com as regras do Despacho nº 49/R/2010
%% Adaptação por João Ferro, Norberto Barroca, Luís Borges, Rui Paulo, Aleksandra Nadziejko - Instituto de Telecomunicações - DEM/UBI, Paulo Machado - Departamento de Ciências Aeroespaciais/UBI.
%% Contacto: latex@e-projects.ubi.pt
%% Agradecimento especial a Stefan_K da latex-community.org pela ajuda com os códigos de tabela e equação.
%% A versão actual pode ser alterada sem aviso prévio.
%% Download da última versão em área reservada: http://www.UBI.pt
%% Uso e distribuição de acordo com a licenca GNU GPL.
%% 
%%    Este programa é um software livre: você pode redistribui-lo e/ou 
%%
%%		modificá-lo dentro dos termos da Licença Pública Geral GNU como 
%%
%%		publicada pela Free Software Foundation, na versão 3 da 
%%
%%		Licença, ou (na sua opinião) qualquer versão.
%%
%%
%%
%%		Este programa é distribuido na esperança que possa ser útil, 
%%
%%		mas SEM NENHUMA GARANTIA; sem uma garantia implícita de ADEQUAÇÃO a qualquer
%%
%%		MERCADO ou APLICAÇÃO EM PARTICULAR. Veja a
%%
%%		Licença Pública Geral GNU para maiores detalhes.
%%
%%
%%
%%		Você deve ter recebido uma cópia da Licença Pública Geral GNU
%%
%%		junto com este programa, se não, veja <http://www.gnu.org/licenses/>.
%%%%%%%%%%%%%%%%%%%%%%%%%%%%%%%%%%%%%%%%%%%%%%%%%%%%%%%%%%%%%%%%%%%%%%%%%%%%%%%%%%%%%%%%%%%%%%%%%%%%%%%%%%%%%%


% Pacotes a incluir
\usepackage{mathspec}
\usepackage{fontspec}
\usepackage{amsmath,amscd,amsthm,xspace}	%Pacotes matemáticos 
\usepackage{amssymb}						%Fontes extra para matemáticos   http://www.ctan.org/tex-archive/fonts/amsfonts
%\usepackage[math]{kurier} %descomentar para colocar tipo de letra aproximado ao Trebuchet nos ambientes matemáticos

%\usepackage[latin1]{inputenc}		%este faz falta na versão normal, mas em XeLaTeX tem que ser comentado		%Permitir caracteres acentuados  http://www.ctan.org/pkg/inputenc
%\usepackage[T1]{fontenc}					%este faz falta na versão normal, mas em XeLaTeX tem que ser comentado		%Permitir caracteresespeciais  http://www.ctan.org/pkg/fontenc

\usepackage[a4paper,left=3.5cm,right=2.5cm,top=2.5cm,bottom=2.5cm]{geometry}	%Papel A4, com margens
%\renewcommand{\baselinestretch}{1.05}

\usepackage{aecompl}								%Permitir fontes vituais para codificação T1  http://www.ctan.org/tex-archive/fonts/ae
\usepackage[center,nooneline,font={footnotesize}]{caption}			%Legenda: centrada, tamanho de nota rodapé  http://www.ctan.org/pkg/caption
\setlength{\parindent}{0pt}							%Sem tabulação em cada novo parágrafo
%\usepackage{parskip}								%Layout with zero \parindent, non-zero \parskip http://www.ctan.org/pkg/parskip
%\setlength{\parskip}{0.53cm}



%% O código seguinte permite gerar um mini indice de capitulo (não referido no despacho reitoral)
% \usepackage[nottoc, notlof, notlot]{tocbibind}
% \usepackage{minitoc}
% \setcounter{minitocdepth}{2}
% \mtcindent=15pt
%%Usar \minitoc para colocar o mini indice de capitulo


%% Verificar se saída é pdf directo e ajustar o formato imagens a isso
%\usepackage{ifpdf}
%\ifpdf
%  \usepackage[pdftex]{graphicx}							%Inserir gráficos  http://ctan.org/pkg/graphicx
\usepackage{graphicx}
%    \DeclareGraphicsExtensions{.jpg,.png} 					%Pdf directo apenas compativel com jpg, png...
  \usepackage[pagebackref,hyperindex=true]{hyperref}				%O pacote hyperref é usada para lidar com comandos referência cruzada  http://ctan.org/pkg/hyperref
%\else
%  \usepackage{graphicx}
%  \DeclareGraphicsExtensions{.ps,.eps}						%DVI directo apenas compativel com ps, eps...
%  \usepackage[dvipdfm,pagebackref,hyperindex=true]{hyperref}
%\fi


\graphicspath{{.}{imagens/}}							%Directorio das imagens


%%Links do pdf
\usepackage{color}								%Pacote de gestão de cor
\definecolor{linkcol}{rgb}{0,0,0} 						%Cor das hiperligações (preto)
\definecolor{citecol}{rgb}{0,0,0} 						%Cor das referências à bibliografia no texto (preto)


%% O código seguinte será incluído no pdf gerado http://www.tug.org/applications/hyperref/manual.html
%%Visto nas propriedades do documento
% \hypersetup
% {
% bookmarksopen=true,
% pdftitle="Tese",		%Título
% pdfauthor="João", 		%Autor
% pdfsubject="Tese", 		%Assunto
% pdfmenubar=true,		%Mostrar barra menus
% pdfhighlight=/O, 		%Efeito ao clicar link
% colorlinks=true, 		%Cor em hiperligações
% pdfpagemode=UseNone, 		%Nenhum modo de páginas
% pdfpagelayout=SinglePage, 	%Abertura em modo de página simples
% pdffitwindow=true, 		%Adaptar página à janela
% linkcolor=linkcol, 		%Cor das ligações internas do documento
% citecolor=citecol, 		%Cor das referências à bibliografia no texto
% urlcolor=linkcol 		%Cor das hiperligações
% }


%%Definições variadas
\setcounter{secnumdepth}{3}
\setcounter{tocdepth}{2}


%%Comandos e atalhos para algumas funções matemáticas
\newcommand{\pd}[2]{\frac{\partial #1}{\partial #2}}
\def\abs{\operatorname{abs}}
\def\argmax{\operatornamewithlimits{arg\,max}}
\def\argmin{\operatornamewithlimits{arg\,min}}
\def\diag{\operatorname{Diag}}
\newcommand{\eqRef}[1]{(\ref{#1})}


%%Para rotação de figuras e tabelas http://www.ctan.org/tex-archive/macros/latex/contrib/rotating
 \usepackage{rotating}
  
	
%%Cabeçalho e rodapé http://www.ctan.org/tex-archive/macros/latex/contrib/fancyhdr
\usepackage{fancyhdr}                   						%Fancy Header and Footer
\pagestyle{fancy}                      							%Cabeçalho e rodapé estilo fancy
\fancyfoot{}                            						%Apaga rodapé actual
\fancyhead{}										%Apaga cabeçalho actual
\fancyfoot[LE,RO]{\thepage}   								%Número de paginas no exterior L=left, E=even, R=right, O=Odd
\newcommand{\cabecalho}[1]{\fancyhead[RE,LO]{\bfseries{#1}}}				%Nome da tese no cabeçalho
\let\headruleORIG\headrule
\renewcommand{\headrule}{\color{black} \headruleORIG}
\renewcommand{\headrulewidth}{0pt}							%Régua para cabecalho, 0pt=off, 1pt=on


%Formatação dos tipos de letra
\newcommand{\capitulos}{\fontsize{22pt}{10pt}\bfseries\selectfont}			%Capítulo: 22pt, negrito
\newcommand{\titulos}{\fontsize{18pt}{10pt}\bfseries\selectfont}			%Titulos: 18pt, negrito
\newcommand{\seccao}{\fontsize{14pt}{20pt}\bfseries\selectfont}				%Secção: 14pt, negrito
\newcommand{\subseccao}{\fontsize{12pt}{14pt}\selectfont}				%subsecção: 12pt, normal
\renewcommand{\footnotesize}{\fontsize{9pt}{12pt}\selectfont}				%Nota rodapé: 9pt, espacamento 1 linha
\renewcommand{\Huge}{\titulos}


%Folha de rosto
\newcommand{\rostoubi}{\fontsize{14pt}{14pt}\selectfont}				%Texto a dizer UBI 14pt, normal
\newcommand{\rostotitulo}{\fontsize{18pt}{18pt}\selectfont}				%Titulo da tese: 18pt, (+negrito)
\newcommand{\rostosubtit}{\fontsize{16pt}{16pt}\selectfont}				%Titulo da tese: 16pt, (+negrito)
\newcommand{\rostonomes}{\fontsize{14pt}{14pt}\selectfont}				%Nome autor, curso: 14pt, (+negrito)
\newcommand{\rostooutros}{\fontsize{12pt}{12pt}\selectfont}				%Local e data: 12pt, (+negrito)
\newcommand{\rostofac}{\fontsize{12.5pt}{12.5pt}\selectfont}				%Faculdade: 12.5pt




%%Passar para português
%\newcommand{\portugues}{\usepackage[portuguese]{babel}		%Descomentar para escrever com regras em Português sem ser em XeLaTeX  http://www.ctan.org/pkg/babel  								% Users of X∃TeX are ad­vised to use poly­glos­sia rather than Ba­bel.

\newcommand{\portugues}{\usepackage{polyglossia} %Comentar para escrever sem ser em modo XeLaTeX http://www.ctan.org/tex-archive/macros/latex/contrib/polyglossia
\setmainlanguage{portuges}                       %Comentar para escrever sem ser em modo XeLaTeX

\addto\captionsportuguese{\renewcommand{\contentsname}{Índice}}
\addto\captionsportuguese{\renewcommand{\indexname}{Índice Remissivo}}}

%linhas das tabelas e afins
\usepackage{colortbl}									%Adicionar cor às tabelas  http://www.ctan.org/tex-archive/macros/latex/contrib/colortbl/
\arrayrulecolor{black}									%Preto


%Estilo plain modificado
\fancypagestyle{plain}{
  %\fancyhead{}	
  %\fancyfoot{}
  \renewcommand{\headrulewidth}{0pt}
}


%%Para algoritmos  http://www.ctan.org/tex-archive/macros/latex/contrib/algorithms/
\usepackage{algorithm}
\usepackage[noend]{algorithmic}


%%Páginas em branco geradas antes de capítulos têm de vir numeradas
\makeatletter

\def\cleardoublepage{\clearpage\if@twoside \ifodd\c@page\else%
  \hbox{}%
  \thispagestyle{plain}%              							%Páginas em branco usam estilo plain
  \newpage%
  \if@twocolumn\hbox{}\newpage\fi\fi\fi}

\makeatother


%%Tabela com 9pt
%\makeatletter										%O código comentado apresentado apresenta problemas
%\renewenvironment{table}{%								%Procurar no forum http://latex-community.org, tópico "Equation Numbering and Table Font Size''
  %\@float{table}\footnotesize}
  %{\end@float}
%\makeatother
\usepackage{etoolbox}									%Para modificar o tamanho de letra na tabela http://ctan.org/pkg/etoolbox
\AtBeginEnvironment{tabular}{\footnotesize}						%Procurar no forum http://latex-community.org, tópico "Equation Numbering and Table Font Size''


%%Número de equação centrado com equação
\makeatletter										%Explicação  http://tex.stackexchange.com/questions/8351/what-do-makeatletter-and-makeatother-do
\def\place@tag{\quad\boxz@}								%Comentar esta linha para alinhar número da equação à direita
\makeatother
\let\equation\align
\let\endequation\endalign
 

%%Código herdado
\newenvironment{maxime}[1]
{
\vspace*{0cm}
\hfill
\begin{minipage}{0.5\textwidth}%
%\rule[0.5ex]{\textwidth}{0.1mm}\\%
\hrulefill $\:$ {\bf #1}\\
%\vspace*{-0.25cm}
\it 
}%
{%

\hrulefill
\vspace*{0.5cm}%
\end{minipage}
}

%mninitoc
%\let\minitocORIG\minitoc
%\renewcommand{\minitoc}{\minitocORIG \vspace{1.5em}}

\usepackage{multirow}					%Create tab­u­lar cells http://www.ctan.org/tex-archive/macros/latex/contrib/multirow
% \usepackage{slashbox}					%Pro­duce tab­u­lar cells with di­ag­o­nal lines in them http://www.ctan.org/pkg/slashbox

\newenvironment{bulletList}				%http://en.wikibooks.org/wiki/LaTeX/List_Structures
{ \begin{list}%
	{$\bullet$}%
	{\setlength{\labelwidth}{25pt}%
	 \setlength{\leftmargin}{30pt}%
	 \setlength{\itemsep}{\parsep}}}%
{ \end{list} }

\newtheorem{definition}{Définition}
\renewcommand{\epsilon}{\varepsilon}

% centered page environment

\newenvironment{vcenterpage}
%{\newpage\vspace*{\fill}\fancyhf{}\renewcommand{\headrulewidth}{0pt}}
{\newpage\vspace*{\fill}\renewcommand{\headrulewidth}{0pt}}
{\vspace*{\fill}\par\pagebreak}

\usepackage{setspace}					%Pro­vides sup­port for set­ting the spac­ing be­tween lines http://www.ctan.org/tex-archive/macros/latex/contrib/setspace
\usepackage{tabularx}					%The pack­age de­fines an en­vi­ron­ment tab­u­larx, an ex­ten­sion of tab­u­lar  http://www.ctan.org/pkg/tabularx
\usepackage{makeidx}					%Index processor  http://www.ctan.org/tex-archive/indexing/makeindex
\newcolumntype{Y}{>{\raggedright\arraybackslash}X}

%% Inicio do bloco
%% Descomentando o este bloco de comandos as  palavras "Figura" e "Tabelas" vão aparecer por extenso nas respectivas Listas
%%
%% Comentado aparece:
%%Lista de Figuras
%%		2.1 Circuito básico com uma fonte de tensão contínua (V) e uma resistência atraves-
%%			sada por uma corrente I. . . . . . . . . . . . . . . . . . . . . . . . . . . . . .3
%%
%% Descomentando aparece
%% 		Figura 2.1 Circuito básico com uma fonte de tensão contínua (V) e uma resistência
%%      atravessada por uma corrente I. . . . . . . . . . . . . . . . . . . . . .3
%%
%\usepackage[titles]{tocloft}
%\newlength{\mylen}
%
%\renewcommand{\cftfigpresnum}{\figurename\enspace}
%\renewcommand{\cftfigaftersnum}{ }
%\settowidth{\mylen}{\cftfigpresnum\cftfigaftersnum}
%\addtolength{\cftfignumwidth}{\mylen}
%
%\renewcommand{\cfttabpresnum}{\tablename\enspace}
%\renewcommand{\cfttabaftersnum}{ }
%\settowidth{\mylen}{\cfttabpresnum\cfttabaftersnum}
%\addtolength{\cfttabnumwidth}{\mylen}
%% Fim do Bloco
