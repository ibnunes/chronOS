\documentclass[10pt,oneside]{estiloUBI}
%%%%%%%%%%%%%%%%%%%%%%%%%%%%%%%%%%%%%%%%%%%%%%%%%%%%%%%%%%%%%%%%%%%%%%%%%%%%%%%%%%%%%%%%%%%%%%%%%%%%%%%%%%%%%%
%% Este é o ficheiro formatacaoUBI.tex - NÃO EDITAR excepto a secção hypersetup!
%% Define a formatação a ser usada em teses apresentadadas na Universidade da Beira Interior, seguindo o despacho Reitoral nº 49/R/2010
%% Versão 2.2 - 01/06/2016 - Podem aparecer as palavras "Figura" e "Tabela" nas respectivas listas
%% Versão 2.1 - 28/03/2014 - Agora compila com o XeLaTeX por causa do tipo de fonte, incluido o estilo de biblipgrafia IEEE, possibilidade de escolha de tipo de fonte matemático
%% Versão 2.0 - 10/11/2011 - Bibliografia agora aparece no índice
%% Versão 1.9 - 10/10/2011 - Resolvido problema em que o texto nas tabelas aparecia em cima da linha superior
%% Versão 1.8 - 12/07/2011 - Legendas são agora centradas
%% Versão 1.7 - 8/07/2011 - Correcção de algumas medidas de acordo com novo modelo de Word
%% Versão 1.6 - 1/07/2011 - Trebuchet inserido como fonte principal
%% Adaptado do original de Oliver Commowick para estar de acordo com as regras do Despacho nº 49/R/2010
%% Adaptação por João Ferro, Norberto Barroca, Luís Borges, Rui Paulo, Aleksandra Nadziejko - Instituto de Telecomunicações - DEM/UBI, Paulo Machado - Departamento de Ciências Aeroespaciais/UBI.
%% Contacto: latex@e-projects.ubi.pt
%% Agradecimento especial a Stefan_K da latex-community.org pela ajuda com os códigos de tabela e equação.
%% A versão actual pode ser alterada sem aviso prévio.
%% Download da última versão em área reservada: http://www.UBI.pt
%% Uso e distribuição de acordo com a licenca GNU GPL.
%% 
%%    Este programa é um software livre: você pode redistribui-lo e/ou 
%%
%%		modificá-lo dentro dos termos da Licença Pública Geral GNU como 
%%
%%		publicada pela Free Software Foundation, na versão 3 da 
%%
%%		Licença, ou (na sua opinião) qualquer versão.
%%
%%
%%
%%		Este programa é distribuido na esperança que possa ser útil, 
%%
%%		mas SEM NENHUMA GARANTIA; sem uma garantia implícita de ADEQUAÇÃO a qualquer
%%
%%		MERCADO ou APLICAÇÃO EM PARTICULAR. Veja a
%%
%%		Licença Pública Geral GNU para maiores detalhes.
%%
%%
%%
%%		Você deve ter recebido uma cópia da Licença Pública Geral GNU
%%
%%		junto com este programa, se não, veja <http://www.gnu.org/licenses/>.
%%%%%%%%%%%%%%%%%%%%%%%%%%%%%%%%%%%%%%%%%%%%%%%%%%%%%%%%%%%%%%%%%%%%%%%%%%%%%%%%%%%%%%%%%%%%%%%%%%%%%%%%%%%%%%


% Pacotes a incluir
\usepackage{mathspec}
\usepackage{fontspec}
\usepackage{amsmath,amscd,amsthm,xspace}	%Pacotes matemáticos 
\usepackage{amssymb}						%Fontes extra para matemáticos   http://www.ctan.org/tex-archive/fonts/amsfonts
%\usepackage[math]{kurier} %descomentar para colocar tipo de letra aproximado ao Trebuchet nos ambientes matemáticos

%\usepackage[latin1]{inputenc}		%este faz falta na versão normal, mas em XeLaTeX tem que ser comentado		%Permitir caracteres acentuados  http://www.ctan.org/pkg/inputenc
%\usepackage[T1]{fontenc}					%este faz falta na versão normal, mas em XeLaTeX tem que ser comentado		%Permitir caracteresespeciais  http://www.ctan.org/pkg/fontenc

\usepackage[a4paper,left=3.5cm,right=2.5cm,top=2.5cm,bottom=2.5cm]{geometry}	%Papel A4, com margens
%\renewcommand{\baselinestretch}{1.05}

\usepackage{aecompl}								%Permitir fontes vituais para codificação T1  http://www.ctan.org/tex-archive/fonts/ae
\usepackage[center,nooneline,font={footnotesize}]{caption}			%Legenda: centrada, tamanho de nota rodapé  http://www.ctan.org/pkg/caption
\setlength{\parindent}{0pt}							%Sem tabulação em cada novo parágrafo
%\usepackage{parskip}								%Layout with zero \parindent, non-zero \parskip http://www.ctan.org/pkg/parskip
%\setlength{\parskip}{0.53cm}



%% O código seguinte permite gerar um mini indice de capitulo (não referido no despacho reitoral)
% \usepackage[nottoc, notlof, notlot]{tocbibind}
% \usepackage{minitoc}
% \setcounter{minitocdepth}{2}
% \mtcindent=15pt
%%Usar \minitoc para colocar o mini indice de capitulo


%% Verificar se saída é pdf directo e ajustar o formato imagens a isso
%\usepackage{ifpdf}
%\ifpdf
%  \usepackage[pdftex]{graphicx}							%Inserir gráficos  http://ctan.org/pkg/graphicx
\usepackage{graphicx}
%    \DeclareGraphicsExtensions{.jpg,.png} 					%Pdf directo apenas compativel com jpg, png...
  \usepackage[pagebackref,hyperindex=true]{hyperref}				%O pacote hyperref é usada para lidar com comandos referência cruzada  http://ctan.org/pkg/hyperref
%\else
%  \usepackage{graphicx}
%  \DeclareGraphicsExtensions{.ps,.eps}						%DVI directo apenas compativel com ps, eps...
%  \usepackage[dvipdfm,pagebackref,hyperindex=true]{hyperref}
%\fi


\graphicspath{{.}{imagens/}}							%Directorio das imagens


%%Links do pdf
\usepackage{color}								%Pacote de gestão de cor
\definecolor{linkcol}{rgb}{0,0,0} 						%Cor das hiperligações (preto)
\definecolor{citecol}{rgb}{0,0,0} 						%Cor das referências à bibliografia no texto (preto)


%% O código seguinte será incluído no pdf gerado http://www.tug.org/applications/hyperref/manual.html
%%Visto nas propriedades do documento
% \hypersetup
% {
% bookmarksopen=true,
% pdftitle="Tese",		%Título
% pdfauthor="João", 		%Autor
% pdfsubject="Tese", 		%Assunto
% pdfmenubar=true,		%Mostrar barra menus
% pdfhighlight=/O, 		%Efeito ao clicar link
% colorlinks=true, 		%Cor em hiperligações
% pdfpagemode=UseNone, 		%Nenhum modo de páginas
% pdfpagelayout=SinglePage, 	%Abertura em modo de página simples
% pdffitwindow=true, 		%Adaptar página à janela
% linkcolor=linkcol, 		%Cor das ligações internas do documento
% citecolor=citecol, 		%Cor das referências à bibliografia no texto
% urlcolor=linkcol 		%Cor das hiperligações
% }


%%Definições variadas
\setcounter{secnumdepth}{3}
\setcounter{tocdepth}{2}


%%Comandos e atalhos para algumas funções matemáticas
\newcommand{\pd}[2]{\frac{\partial #1}{\partial #2}}
\def\abs{\operatorname{abs}}
\def\argmax{\operatornamewithlimits{arg\,max}}
\def\argmin{\operatornamewithlimits{arg\,min}}
\def\diag{\operatorname{Diag}}
\newcommand{\eqRef}[1]{(\ref{#1})}


%%Para rotação de figuras e tabelas http://www.ctan.org/tex-archive/macros/latex/contrib/rotating
 \usepackage{rotating}
  
	
%%Cabeçalho e rodapé http://www.ctan.org/tex-archive/macros/latex/contrib/fancyhdr
\usepackage{fancyhdr}                   						%Fancy Header and Footer
\pagestyle{fancy}                      							%Cabeçalho e rodapé estilo fancy
\fancyfoot{}                            						%Apaga rodapé actual
\fancyhead{}										%Apaga cabeçalho actual
\fancyfoot[LE,RO]{\thepage}   								%Número de paginas no exterior L=left, E=even, R=right, O=Odd
\newcommand{\cabecalho}[1]{\fancyhead[RE,LO]{\bfseries{#1}}}				%Nome da tese no cabeçalho
\let\headruleORIG\headrule
\renewcommand{\headrule}{\color{black} \headruleORIG}
\renewcommand{\headrulewidth}{0pt}							%Régua para cabecalho, 0pt=off, 1pt=on


%Formatação dos tipos de letra
\newcommand{\capitulos}{\fontsize{22pt}{10pt}\bfseries\selectfont}			%Capítulo: 22pt, negrito
\newcommand{\titulos}{\fontsize{18pt}{10pt}\bfseries\selectfont}			%Titulos: 18pt, negrito
\newcommand{\seccao}{\fontsize{14pt}{20pt}\bfseries\selectfont}				%Secção: 14pt, negrito
\newcommand{\subseccao}{\fontsize{12pt}{14pt}\selectfont}				%subsecção: 12pt, normal
\renewcommand{\footnotesize}{\fontsize{9pt}{12pt}\selectfont}				%Nota rodapé: 9pt, espacamento 1 linha
\renewcommand{\Huge}{\titulos}


%Folha de rosto
\newcommand{\rostoubi}{\fontsize{14pt}{14pt}\selectfont}				%Texto a dizer UBI 14pt, normal
\newcommand{\rostotitulo}{\fontsize{18pt}{18pt}\selectfont}				%Titulo da tese: 18pt, (+negrito)
\newcommand{\rostosubtit}{\fontsize{16pt}{16pt}\selectfont}				%Titulo da tese: 16pt, (+negrito)
\newcommand{\rostonomes}{\fontsize{14pt}{14pt}\selectfont}				%Nome autor, curso: 14pt, (+negrito)
\newcommand{\rostooutros}{\fontsize{12pt}{12pt}\selectfont}				%Local e data: 12pt, (+negrito)
\newcommand{\rostofac}{\fontsize{12.5pt}{12.5pt}\selectfont}				%Faculdade: 12.5pt




%%Passar para português
%\newcommand{\portugues}{\usepackage[portuguese]{babel}		%Descomentar para escrever com regras em Português sem ser em XeLaTeX  http://www.ctan.org/pkg/babel  								% Users of X∃TeX are ad­vised to use poly­glos­sia rather than Ba­bel.

\newcommand{\portugues}{\usepackage{polyglossia} %Comentar para escrever sem ser em modo XeLaTeX http://www.ctan.org/tex-archive/macros/latex/contrib/polyglossia
\setmainlanguage{portuges}                       %Comentar para escrever sem ser em modo XeLaTeX

\addto\captionsportuguese{\renewcommand{\contentsname}{Índice}}
\addto\captionsportuguese{\renewcommand{\indexname}{Índice Remissivo}}}

%linhas das tabelas e afins
\usepackage{colortbl}									%Adicionar cor às tabelas  http://www.ctan.org/tex-archive/macros/latex/contrib/colortbl/
\arrayrulecolor{black}									%Preto


%Estilo plain modificado
\fancypagestyle{plain}{
  %\fancyhead{}	
  %\fancyfoot{}
  \renewcommand{\headrulewidth}{0pt}
}


%%Para algoritmos  http://www.ctan.org/tex-archive/macros/latex/contrib/algorithms/
\usepackage{algorithm}
\usepackage[noend]{algorithmic}


%%Páginas em branco geradas antes de capítulos têm de vir numeradas
\makeatletter

\def\cleardoublepage{\clearpage\if@twoside \ifodd\c@page\else%
  \hbox{}%
  \thispagestyle{plain}%              							%Páginas em branco usam estilo plain
  \newpage%
  \if@twocolumn\hbox{}\newpage\fi\fi\fi}

\makeatother


%%Tabela com 9pt
%\makeatletter										%O código comentado apresentado apresenta problemas
%\renewenvironment{table}{%								%Procurar no forum http://latex-community.org, tópico "Equation Numbering and Table Font Size''
  %\@float{table}\footnotesize}
  %{\end@float}
%\makeatother
\usepackage{etoolbox}									%Para modificar o tamanho de letra na tabela http://ctan.org/pkg/etoolbox
\AtBeginEnvironment{tabular}{\footnotesize}						%Procurar no forum http://latex-community.org, tópico "Equation Numbering and Table Font Size''


%%Número de equação centrado com equação
\makeatletter										%Explicação  http://tex.stackexchange.com/questions/8351/what-do-makeatletter-and-makeatother-do
\def\place@tag{\quad\boxz@}								%Comentar esta linha para alinhar número da equação à direita
\makeatother
\let\equation\align
\let\endequation\endalign
 

%%Código herdado
\newenvironment{maxime}[1]
{
\vspace*{0cm}
\hfill
\begin{minipage}{0.5\textwidth}%
%\rule[0.5ex]{\textwidth}{0.1mm}\\%
\hrulefill $\:$ {\bf #1}\\
%\vspace*{-0.25cm}
\it 
}%
{%

\hrulefill
\vspace*{0.5cm}%
\end{minipage}
}

%mninitoc
%\let\minitocORIG\minitoc
%\renewcommand{\minitoc}{\minitocORIG \vspace{1.5em}}

\usepackage{multirow}					%Create tab­u­lar cells http://www.ctan.org/tex-archive/macros/latex/contrib/multirow
% \usepackage{slashbox}					%Pro­duce tab­u­lar cells with di­ag­o­nal lines in them http://www.ctan.org/pkg/slashbox

\newenvironment{bulletList}				%http://en.wikibooks.org/wiki/LaTeX/List_Structures
{ \begin{list}%
	{$\bullet$}%
	{\setlength{\labelwidth}{25pt}%
	 \setlength{\leftmargin}{30pt}%
	 \setlength{\itemsep}{\parsep}}}%
{ \end{list} }

\newtheorem{definition}{Définition}
\renewcommand{\epsilon}{\varepsilon}

% centered page environment

\newenvironment{vcenterpage}
%{\newpage\vspace*{\fill}\fancyhf{}\renewcommand{\headrulewidth}{0pt}}
{\newpage\vspace*{\fill}\renewcommand{\headrulewidth}{0pt}}
{\vspace*{\fill}\par\pagebreak}

\usepackage{setspace}					%Pro­vides sup­port for set­ting the spac­ing be­tween lines http://www.ctan.org/tex-archive/macros/latex/contrib/setspace
\usepackage{tabularx}					%The pack­age de­fines an en­vi­ron­ment tab­u­larx, an ex­ten­sion of tab­u­lar  http://www.ctan.org/pkg/tabularx
\usepackage{makeidx}					%Index processor  http://www.ctan.org/tex-archive/indexing/makeindex
\newcolumntype{Y}{>{\raggedright\arraybackslash}X}

%% Inicio do bloco
%% Descomentando o este bloco de comandos as  palavras "Figura" e "Tabelas" vão aparecer por extenso nas respectivas Listas
%%
%% Comentado aparece:
%%Lista de Figuras
%%		2.1 Circuito básico com uma fonte de tensão contínua (V) e uma resistência atraves-
%%			sada por uma corrente I. . . . . . . . . . . . . . . . . . . . . . . . . . . . . .3
%%
%% Descomentando aparece
%% 		Figura 2.1 Circuito básico com uma fonte de tensão contínua (V) e uma resistência
%%      atravessada por uma corrente I. . . . . . . . . . . . . . . . . . . . . .3
%%
%\usepackage[titles]{tocloft}
%\newlength{\mylen}
%
%\renewcommand{\cftfigpresnum}{\figurename\enspace}
%\renewcommand{\cftfigaftersnum}{ }
%\settowidth{\mylen}{\cftfigpresnum\cftfigaftersnum}
%\addtolength{\cftfignumwidth}{\mylen}
%
%\renewcommand{\cfttabpresnum}{\tablename\enspace}
%\renewcommand{\cfttabaftersnum}{ }
%\settowidth{\mylen}{\cfttabpresnum\cfttabaftersnum}
%\addtolength{\cfttabnumwidth}{\mylen}
%% Fim do Bloco


%\usepackage[english,portuges]{babel}
%\usepackage[utf8]{inputenc}
%\usepackage[T1]{fontenc}
\usepackage{siunitx}
%\usepackage{graphicx}
\usepackage{minted}
\usepackage{tikz}
\usepackage{pgfplots}
%\usepackage{hyperref}
\usepackage[printonlyused]{acronym}

%\graphicspath{{img/}}

\pgfplotsset{compat = newest}

\newcommand{\chronOS}{\textsf{chronOS}}
\newcommand{\version}{1.3.0}
\newcommand{\git}{\textsf{git}}

\newcounter{note}
\newenvironment{note}[1][]
	{
		\medskip \refstepcounter{note} \par \medskip
		\noindent\textcolor[RGB]{220,220,220}{\rule{\linewidth}{0.4pt}}
		\noindent \textbf{Nota~\thenote. #1} \rmfamily
	}
	{
		\\ \noindent\textcolor[RGB]{220,220,220}{\rule{\linewidth}{0.4pt}} \medskip
	}


%\title{
%	Sistemas Operativos --- Projeto\\
%	\chronOS~--- \textbf{Relatório final}
%}
%
%\author{
%	\begin{tabular}[!h]{l l}
%		39489 & Jorge Miguel Louro Pissarra\\
%		41266 & Diogo Castanheira Simões\\
%		41381 & Igor Cordeiro Bordalo Nunes
%	\end{tabular}
%}
%
%\date{\today}

\portugues
\makeindex

\cabecalho{\chronOS~\version}

\begin{document}
	\onehalfspacing
	\pagenumbering{roman}
	\begin{titlepage}
\begin{center}

\begin{flushleft}
\includegraphics[height=2.22cm]{logo}\\
\rostoubi UNIVERSIDADE DA BEIRA INTERIOR\\
\rostofac Departamento de Informática\\
\end{flushleft}

%\vspace{7.6cm}
\vspace{1.5cm}

\begin{center}
\includegraphics[height=5cm]{chronos.png}
\end{center}

\rostotitulo \textbf{\chronOS~1.2.0} \\
\rostosubtit \textit{A scheduling simulator}\\

\vspace{1.8cm}

\begin{tabular}{>{\rostonomes\bfseries}l @{\rostonomes\bfseries~---~} >{\rostonomes\bfseries}l}
	39489 & Jorge Miguel Louro Pissarra \\
	41266 & Diogo Castanheira Simões \\
	41381 & Igor Cordeiro Bordalo Nunes \\
\end{tabular}

\vspace{1.4cm}

\rostooutros Sistemas Operativos\\
\rostonomes \textbf{Engenharia Informática}\\
\rostooutros (1º ciclo de estudos)\\

\vspace{2.1cm}

\rostooutros Docente: Prof. Doutor Paul Andrew Crocker\\
% \rostooutros Orientador: Prof. Doutor João Paulo da Costa Cordeiro\\
%Co-orientador: Prof. Doutor Nome\\

\vspace{1.2cm}

\rostooutros \textbf{Covilhã, 14 de junho de 2020}

\end{center}
\end{titlepage}


	
	%\dominitoc
	
	\pagestyle{fancy}
	
	%\cleardoublepage
	\newpage
	
	\section*{\titulos{Resumo}}
	\vspace{0.5cm}
	
	\chronOS~é um programa de simulação de escalonamento de processos e gestão de memória.
	
	Estão implementados:
	
	\begin{itemize}
		\item Os algoritmos \ac{FCFS}, \ac{SJF}, \textit{round-robin} e \ac{PSA} para o escalonamento;
		\item Um escalonador de longo prazo;
		\item O algoritmo \textit{first-fit} para a gestão de memória estática;
		\item Os algoritmod \textit{first-fit}, \textit{next-fit}, \textit{best-fit} e \textit{worst-fit} para a gestão de memória dinâmica.
	\end{itemize}
	
	O gestor de memória dinâmica (\textit{heap memory}) foi implementado recorrendo a listas ligadas. Para cada um dos 4 algoritmos implementados para a sua gestão foi criada uma memória dedicada para fins estatísticos e comparativos.
	
	Um plano de execução é lido no arranque do programa e o simulador de CPU, definido para \SI{2}{\hertz}, invoca o método que implementa o algoritmo de escalonamento para decidir a próxima ação, a qual pode ser um \textit{switch} no estado de um processo ou a execução de uma instrução.
	
	O programa \chronOS~tem a capacidade de fazer pedidos pseudoaleatórios de alocação de memória dinâmica a fim de melhor testar a respetiva gestão.
	
	Vários parâmetros do \chronOS~podem ser alterados com argumentos passados pela linha de comandos, como por exemplo a seleção do algoritmo de escalonamento e a semente do gerador pseudoaleatório de solicitações de alocação de memória dinâmica do programa.
	
	Sempre que possível é utilizada a memória \textit{heap} do computador para albergar os componentes do programa. Tal inclui o bloco de memória estática, a tabela PCB e as quatro memórias dinâmicas.
	
	O \chronOS~funciona segundo o princípio de uma \textit{state machine}. Uma estrutura global com diversas variáveis e \textit{flags} é utilizada para que o programa regule o seu fluxo de execução e outros parâmetros.
	
	No presente relatório é apresentada a versão \version.

	
	%\cleardoublepage
	\tableofcontents
	\listoffigures
	%\cleardoublepage	
	\mainmatter


	
	\chapter{Pesquisa e desenvolvimento}
	\label{sec:dev}
	
	Conhecer o coração dos sistemas operativos modernos implica conhecer diversos métodos de gestão de memória e de processos. \chronOS~propõe-se enquanto projeto de simulação de escalonamento e gestão de memória a fim de consolidar conhecimentos na área no âmbito da unidade curricular de Sistemas Operativos.
	
	\section{Linguagem e pesquisa}
	\label{ssec:dev:sota}
	
	Para este projeto foram consideradas as linguagens C, C++ e OCaml. Optou-se pela linguagem C devido à facilidade em se trabalhar a um mais baixo nível e à grande versatilidade que os apontadores permitem. C++ teria sido uma linguagem ideal para usar orientação a objetos, mas a pouca familiaridade com esta levou-nos a optar pelo C.
	
	Para reforçar a nossa aprendizagem, decidimos implementar tudo ``\textit{in-house}'', \textbf{não} recorrendo a códigos já existentes ou projetos semelhantes na Internet e em repositórios \git~públicos.
	
	
	\section{Repositório no GitLab}
	\label{ssec:dev:gitlab}
	
	O projeto encontra-se alojado num servidor GitLab privado no seguinte \textit{link}: \url{https://gitlab.pcdev.pt/inunes/chronos/}
	
	O repositório \git~em causa inclui vários \textit{branches} nos quais temos trabalhado, a saber:
	
	\begin{itemize}
		\item \texttt{master}: \textit{branch} principal para onde é feito \textit{merge} de versões finais funcionais;
		
		\item \texttt{dev}: \textit{branch} de desevolvimento dedicado do elemento de grupo Jorge Pissarra;
		
		\item \texttt{dev-in}: \textit{branch} de desevolvimento dedicado do elemento de grupo Igor Nunes (regra geral é neste \textit{branch} que se encontram as \textbf{versões compiláveis sem erros com recurso ao \texttt{Makefile}});
		
		\item \texttt{dev-ds}: \textit{branch} de desevolvimento dedicado do elemento de grupo Diogo Simões;
		
		\item \texttt{dev-sched}: \textit{branch} temporário de desenvolvimento do elemento de grupo Diogo Simões para implementar algoritmos de escalonamento. O algoritmo \ac{SJF} por ele implementado foi adotado na versão final, apesar de nenhum \textit{merge} ter sido feito no repositório \git~\textit{per se}.
	\end{itemize}

	Para as versões finais foram criadas \textit{tags} numeradas com a versão correspondente do \chronOS.
	
	A licença adotada para o projeto é a \textit{GNU General Public License version 3}.
	

	\section{Resumo do projeto}
	\label{ssec:dev:summary}
	
	À data de escrita do presente relatório, o programa \chronOS~é capaz de fazer as seguintes tarefas:
	
	\begin{enumerate}
		\item Alocar células de memória e a tabela \texttt{\ac{PCB}};
		\item Alocar 4 memórias \textit{heap} (uma por algoritmo de gestão);
		\item Ler o ficheiro \texttt{plan.txt} e guardar em memória o plano de execução numa \textit{queue}, incluindo a prioridade de cada processo;
		\item Ler o ficheiro \texttt{control.txt} e guardar em memória o plano de controlo da aplicação;
		\item Ser controlado pelo ficheiro, manual ou automaticamente por opção;
		\item Temporizar o sistema de 500 em 500 milissegundos;
		\item Extrair de um ficheiro \texttt{*.prg} as suas instruções;
		\item Alocar em memória as instruções referentes a um programa recorrendo ao \textbf{algoritmo \textit{first-fit}};
		\item Ler um programa para memória e alocar os dados na tabela \ac{PCB} ao tempo exato indicado pelo plano de execução;
		\item Executar as instruções em memória, incluindo o \textit{fork} (instrução \verb|C n|) e o \textit{clean} (instrução \verb|L filename|);
		\item Gerir os processos com recurso a um dos seguintes algoritmos: \textbf{\ac{FCFS}}, \textbf{\ac{SJF}}, \textbf{\textit{round-robin}} ou \textbf{\ac{PSA}};
		\item Executar o escalonador de longo prazo;
		\item Alocar e dealocar memória \textit{heap} a pedido de cada processo;
		\item Alocar memória \textit{heap} por solicitação pseudoaleatória do processo principal;
		\item Imprimir um relatório do estado dos processos;
		\item Imprimir um relatório do estado da memória em modo \textit{debug};
		\item Imprimir um relatório das memórias \textit{heap};
		\item Fazer um \textit{dump} das memórias \textit{heap} em modo \textit{debug};
		\item Libertar os recursos associados às memórias, à tabela \ac{PCB} e à \textit{queue} do plano de execução.
	\end{enumerate}
	
	Todas as tarefas foram testadas em modo \textit{debug} recorrendo a \textit{sanitizer flags} (descritas na secção \ref{ssec:dev:struct_makefile}) que nos permite verificar que não existe nenhum \textit{memory leak} no final da execução do programa.
	
	
	\section{Estrutura do código e \texttt{Makefile}}
	\label{ssec:dev:struct_makefile}
	
	A pasta de desenvolvimento está estruturada com as seguintes pastas:
	
	\begin{itemize}
		\item \verb|doc|: pasta onde é escrita a documentação, incluindo este mesmo documento em \LaTeX;
		\item \verb|include|: inclui todos os \textit{header files} (ficheiros \verb|*.h|) necessários ao programa;
		\item \verb|src|: inclui todos os ficheiros \verb|*.c| com a implementação dos protótipos declarados nos \textit{header files};
		\item \verb|obj|: inclui os ficheiros binários \verb|*.o| gerados pelo processo de compilação;
		\item \verb|bin|: pasta onde é gerado o ficheiro executável \verb|chronos|.
	\end{itemize}

	Por sua vez, o \texttt{Makefile} está construído de forma a possibilitar 2 formas de compilação:
	
	\begin{itemize}
		\item \textbf{\textit{Debug}} (\verb|make debug|): versão de desenvolvimento com mensagens de \textit{debugging} bastante detalhadas;
		
		\item \textbf{\textit{Release}} (\verb|make release|): versão final sem mensagens de \textit{debug} durante a execução do \chronOS.
	\end{itemize}

	O \texttt{Makefile} inclui uma série de \textit{flags} adicionais face ao comum a fim de controlar de forma mais apertada o comportamento do \ac{gcc}, o qual é bastante permissivo por natureza. O modo \textit{debug} inclui \textit{sanitizer flags} que permitem detetar \textit{memory leaks} e detalhar informações acerca de falhas graves na execução relacionadas com acessos indevidos à memória (os clássicos \textit{segmentation faults}).
	
	
	\chapter{Execução do \chronOS}
	\label{sec:main}
	
	O programa começa por alocar todos os recursos necessários à execução do programa, nomeadamente:
	
	\begin{itemize}
		\item \verb|memory|: bloco de memória, descrito na secção \ref{sec:memory}, constituída por 1000 células;
		\item \verb|pcb|: tabela \ac{PCB}, descrita na secção \ref{sec:process}, constituída por 100 linhas;
		\item \verb|plan|: \textit{queue} com o plano de entrada de processos no sistema, dinamicamente alocada e ajustada;
		\item \verb|heap_*|: memórias \textit{heap}, constituídas por 128 partições de 2 KB cada, uma por cada algoritmo de gestão implementado, a saber:
		\begin{enumerate}
			\item \verb|heap_first|: reservada para o algoritmo \textit{first-fit};
			\item \verb|heap_next|: reservada para o algoritmo \textit{next-fit};
			\item \verb|heap_best|: reservada para o algoritmo \textit{best-fit};
			\item \verb|heap_worst|: reservada para o algoritmo \textit{worst-fit}.
		\end{enumerate}
	\end{itemize}

	O plano de entrada de processos é lido a partir do ficheiro \texttt{plan.txt}, o qual deve estar obrigatoriamente presente. O plano é alocado em \textit{heap memory} numa \textit{queue} do tipo \verb|plan_q|, definida em \verb|plan.h|. É esperado que o ficheiro tenha os dados ordenados por ordem cronológica de entrada. O comportamento do \chronOS~ para um plano não ordenado não está definido.
	
	\begin{note}
		O programa determina automaticamente qual a pasta onde o executável se encontra a fim de aí encontrar os ficheiros \verb|plan.txt| e os programas de extensão \verb|*.prg|.

		Tal permite que o programa possa ser invocado a partir de qualquer diretório.
	\end{note}
	
	\chronOS~executa com base num ciclo que simula uma \ac{CPU} com um \textit{clock} de \SI{2}{\hertz}. A cada ciclo de \textit{clock} é considerado que passou 1 unidade de tempo de \ac{CPU}, armazenada em \verb|cputime|, e que permite determinar os tempos aos quais os processos entram em execução, quanto tempo levam a ser executados e em que tempo são terminados.
	
	A cada ciclo de \textit{clock}, um algoritmo de escalonamento é executado, o qual irá definir a próxima ação a ser tomada. Os algoritmos de escalonamento têm o controlo \textit{de facto} da execução das instruções e dos processos, não sendo responsabilidade da função \verb|main| onde se encontra o simulador da \ac{CPU}.
	
	\begin{figure}[!btp]
		\centering
		\includegraphics[width=\textwidth]{pcbreport}
		\caption{Exemplo de um relatório da tabela de processos.}
		\label{fig:pcbreport}
	\end{figure}
	
	\begin{figure}[!btp]
		\centering
		\includegraphics[scale=0.6]{memreport}
		\caption{Exemplo de um relatório da memória estática em modo \textit{debug}.}
		\label{fig:memreport}
	\end{figure}

	Também é determinado por um método pseudoaleatório se o processo principal do \chronOS~deverá alocar memória dinâmica e, em caso positivo, quantos blocos deverão ser alocados, entre o mínimo e o máximo permitidos (i.e., entre 3 e 10). Este processo será explorado em mais detalhe no Capítulo \ref{sec:heap}, assim como os algoritmos de gestão de memória implementados.

	Para terminar, todos os recursos são libertados e é impresso um relatório da tabela \ac{PCB} (Figura \ref{fig:pcbreport}) e um das memórias dinâmicas (Figura \ref{fig:heapreport}). Em modo de \textit{debug}, e exclusivamente para efeitos de \textit{debugging} no processo de desenvolvimento, é impresso um relatório adicional da memória estática (Figura \ref{fig:memreport}) e \textit{dumps} das 4 memórias dinâmicas (Figura \ref{fig:heapdump}).
	
	
	\section{\textit{State machine}. Argumentos passados pela linha de comandos.}
	\label{ssec:main:argv}
	
	\chronOS~admite passagem opcional de argumentos pela linha de comandos a fim de modificar a configuração padrão do programa, definida com a estrutura \mintinline{c}{struct world w} (definida em \verb|types.h|, declarada em \verb|data.h| e inicializada em \verb|world.c|). Esta estrutura permite que o \chronOS~funcione segundo o princípio de uma \textit{\textbf{state machine}}, albergando diversos valores e \textit{flags} que o programa irá utilizar para controlar o seu fluxo e alguns algoritmos de escalonamento (em particular o \textit{round-robin}, discutido na secção \ref{ssec:process:rrobin}).
	
	A sintaxe para chamada do programa é a seguinte:
	
	\begin{verbatim}
./chronos
   [<--fcfs | --sjf | <--rr | --robin | --round-robin> | --psa>]
   [<<--no-heap-request | -n> | <--seed | -s> n>]
   [<--control | -c> <auto | stdin | file>]
	\end{verbatim}
	
	\begin{enumerate}
		\item Definição do algoritmo de escalonamento:
		\begin{itemize}
			\item \verb|--fcfs|: algoritmo \ac{FCFS} (algoritmo \textbf{definido por defeito});
			\item \verb|--sjf|: algoritmo \ac{SJF};
			\item \verb|--rr|, \verb|--robin| ou \verb|--round-robin|: algoritmo \textit{round-robin};
			\item \verb|--psa|: algoritmo \ac{PSA}.
		\end{itemize}
	
		\item Definição da semente de solicitações pseudoaleatórias da memória \textit{heap}:
		\begin{itemize}
			\item \verb|--no-heap-request| ou \verb|-n|: desliga os requisitos pseudoaleatórios de memória \textit{heap} por parte do \chronOS;
			\item \verb|--seed n| ou \verb|-s n|: define semente dada pelo número inteiro \verb|n|. Caso seja fornecido um número fraccionário é admitida a parte inteira. Em caso de não ser um número é definido 0 (zero). O \textbf{valor por defeito} é definido pelo tempo (descrito na secção \ref{ssec:heap:request}).
		\end{itemize}
	
		\item Definição do controlador da aplicação (\verb|--control| ou \verb|-c|):
		\begin{itemize}
			\item \verb|-c auto|: modo automático, onde o programa procura terminar todos os processos até ao fim se possível (este é o \textbf{modo por defeito});
			\item \verb|-c stdin|: modo manual, onde o programa é controlado pelo utilizador;
			\item \verb|-c file|: modo por ficheiro, onde o programa é controlado segundo as instruções indicadas pelo ficheiro \verb|control.txt|.
		\end{itemize}
	\end{enumerate}
	
	Chamadas válidas do \chronOS~incluem:
	
	\begin{verbatim}
./chronos
./chronos --seed 2020
./chronos --fcfs -s 15
./chronos --sjf -n
./chronos --seed 42 --control file
	\end{verbatim}
	
	
	
	\chapter{Gestão de processos}
	\label{sec:process}
	
	Um \textbf{processo} inclui as seguintes informações, definidas na estrutura \verb|process| no \textit{header file} \texttt{types.h}:
	
	\begin{itemize}
		\item \verb|name|: nome do processo;
		\item \verb|pid|: \ac{PID} do processo;
		\item \verb|ppid|: \ac{PID} do processo pai;
		\item \verb|context|: variável inteira associada ao processo;
		\item \verb|start|: endereço de memória da 1ª instrução do processo;
		\item \verb|counter|: PC [\textit{Program Counter}] do processo (endereço absoluto no bloco de memória);
		\item \verb|instsize|: número de instruções do processo;
		\item \verb|state|: estado atual do processo;
		\item \verb|priority|: nível de prioridade do processo;
		\item \verb|timelimit|: \textit{burst time} do processo;
		\item \verb|timeinit|: tempo de \ac{CPU} ao qual o processo iniciou (passou ao estado \texttt{STATUS\_READY});
		\item \verb|timeend|: tempo de \ac{CPU} ao qual o processo terminou (passou ao estado \texttt{STATUS\_TERMINATED});
		\item \verb|timeused|: tempo total consumido a ser processado (durante o estado \texttt{STATUS\_RUNNING}).
	\end{itemize}
	
	Neste momento, \chronOS~implementa o modelo de 5 estados de processos, controlado por um método \verb|switchState|. Este método será utilizado por todos os algoritmos de gestão e escalonamento que serão implementados.
	
	Da mesma forma que a gestão de memória, falada na secção \ref{sec:memory}, a gestão de processos está dividida em três componentes:
	
	\begin{itemize} %Create an acronym session to store this
	    \item O \texttt{\ac{PCB}}, que representa a estrutura onde estão armazenados todos os processos;
	    \item As funções definidas no ficheiro \texttt{\ac{PCB}}, que servem para manipular o \texttt{\ac{PCB}};
	    \item Os algoritmos de escalonamento.
	\end{itemize}  
	
	É, portanto, possível efetuar as atuais operações no \ac{PCB}:
	\begin{itemize}
	    \item Criar uma tabela \ac{PCB};
	    % \item Obter o \ac{PID} mais elevado presente;
	    \item Alocar um processo e armazená-lo na tabela \ac{PCB};
	    \item Libertar toda a estrutura;
	    \item Obter a informação de um processo na estrutura dado o seu \ac{PID}.
	\end{itemize}
	
	Os algoritmos de escalonamento implementados na atual \textit{build} do programa são: \ac{FCFS}, \ac{SJF}, \textit{round-robin} e \ac{PSA}. Adiciona-se ainda o escalonador de longo prazo.
	
	
	\section{Algoritmo \ac{FCFS}}
	\label{ssec:process:fcfs}
	
	O \textbf{algoritmo \ac{FCFS}} admite os novos processos (estado \texttt{STATUS\_NEW}) na fila de processos prontos (estado \texttt{STATUS\_READY}) e, uma vez no estado pronto, faz o seu \textit{dispatch} para a fila de processos em execução (estado \texttt{STATUS\_RUNNING}). Uma vez terminada a sua execução, é feito o \textit{release} e o processo fica na fila de processos terminados (estado \texttt{STATUS\_TERMINATED}).
	
	Tudo é feito segundo o princípio FIFO [\textit{First In, First Out}], i.e., uma \textit{queue}. Dada a forma como a estrutura \ac{PCB} está definida no nosso projeto, o método \verb|fcfs| necessita apenas de saber o índice atual da tabela \ac{PCB} a fim de saber as próximas ações a tomar segundo o modelo de 5 estados.
	
	
	\section{Algoritmo \ac{SJF}}
	\label{ssec::process:sjf}
	
	De forma semelhante ao \ac{FCFS}, o algoritmo \ac{SJF} promove o \textit{switch} de estados dos processos, sem capacidade de desbloquear os processos em estado \verb|STATUS_BLOCKED|.
	
	A estratégia adotada baseou-se no \textit{sorting} na tabela \ac{PCB} segundo o \textit{burst time} de cada processo (com recurso à função \verb|qsort()|). Tal simplificou a implementação da função que alberga o algoritmo, \verb|sjf|. Para ter em atenção a \textit{performance}, a tabela apenas é ordenada caso ela não esteja por ordem crescente em relação ao tamanho dos processos.
	
	Para fins de teste e \textit{debugging} do algoritmo, foi definido que o \textit{burst time} é igual ao número de instruções do processo.
	
	
	\section{Algoritmo \textit{round-robin}}
	\label{ssec:process:rrobin}
	
	Este algoritmo recorre à \textit{state machine}, apresentada na secção \ref{ssec:main:argv}), para o seu pleno funcionamento. A estrutura global \mintinline{c}{struct world w} referente à \textit{state machine} inclui o campo \verb|w.heap.rr_time| que armazena quantos \textit{ticks} de \textit{clock} do \chronOS~passaram a fim de se poder saber quando é feito o \textit{switch} do estado do processo atual para \verb|STATUS_READY|. Com este \textit{switch} é feito o \textit{reset} a esta variável e um novo processo é executado.
	
	Quando se alcança o fim da tabela \ac{PCB} o índice é reiniciado a zero. Contudo, em todas as chamadas à função \verb|rrobin| que alberga o algoritmo, a função \verb|checkPCBStatus| é invocada e esta verifica se todos os processos estão terminados ou bloqueados. Em caso positivo é devolvido à função principal o valor \verb|SCHEDULER_END|, o qual indica o fim do processo de escalonamento.
	
	
	\section{Algoritmo \ac{PSA}}
	\label{ssec:process:psa}
	
	Este algoritmo aplica uma estratégia semelhante ao \ac{SJF} a nível de implementação. Contudo, os processos são organizados por prioridade a fim de os mais prioritários serem executados em primeiro lugar.
	
	
	\section{Escalonador de longo prazo}
	\label{ssec:process:longterm}
	
	O escalonador de longo prazo é o responsável por desbloquear processos que estejam presos no estado \verb|STATUS_BLOCKED|. Os escalonadores apresentados até agora não têm a capacidade de desbloquear estes processos, sendo portanto responsabilidade deste escalonador promover o respetivo desbloqueio. Este é chamado pela função principal com alguma regularidade, manualmente por indicação do utilizador através do \verb|stdin|, ou segundo as instruções do ficheiro \verb|control.txt|.
	
	A função \verb|ltsched| é a responsável por este processo e é invocada por via de controlo manual ou por ficheiro da aplicação (secção \ref{ssec:process:control}).
	
	
	\section{Controlo da aplicação}
	\label{ssec:process:control}
	
	Existem 3 modos de controlar o \chronOS:
	
	\begin{enumerate}
		\item \textbf{Automático}: A aplicação tenta terminar todos os processos antes de encerrar. Um relatório é impresso apenas no final da execução de todos os processos.
		
		\item \textbf{Manual}: O utilizador é questionado para controlar a aplicação inserindo as instruções. Contudo, é possível modificar este comportamento usando um redirecionamento do \verb|stdin| aquando da chamada do \chronOS.
		
		\item \textbf{Por ficheiro}: A aplicação procura o ficheiro \verb|control.txt|, carrega as instruções em memória numa \textit{queue} e executa as instruções em sequência.
	\end{enumerate}

	
	\begin{note}
		Existe uma limitação na versão \version.
		
		Na ausência de mais instruções para executar, e não estando a \textit{queue} de planificação vazia, só é feito um \textit{fetch} do próximo comando de controlo quando o próximo processo na fila é carregado em memória.
	\end{note}
	
	As instruções são as seguintes:
	
	\begin{itemize}
		\item \verb|E|: Executar por \textbf{5} unidades de tempo;
		\item \verb|I|: Bloquear o processo atual;
		\item \verb|D|: Executar o escalonador de longo prazo;
		\item \verb|R|: Imprimir um relatório;
		\item \verb|T|: Terminar forçosamente a execução do \chronOS.
	\end{itemize}
	
	
	\chapter{Gestão de memória estática}
	\label{sec:memory}
	
	A memória no \chronOS~é, na sua essência, um bloco monolítico constituído células de memória. Cada célula é uma instrução. A estrutura \verb|MEMORY| definida em \texttt{types.h} define este bloco.
	
	A gestão de memória está dividida em duas componentes:
	\begin{itemize}
		\item A \verb|memory|, que representa a estrutura de memória onde são armazenadas as instruções;
		\item As funções definidas no ficheiro \verb|memmgr.c|, que servem para manipular a \verb|memory|.
	\end{itemize}
	
	Na versão do projeto aqui apresentada, é possível efetuar as seguintes operações sobre a \verb|memory|:
	\begin{itemize}
		\item Criar um bloco de memória (\verb|memcreate|);
		\item Limpar instruções na memória (\verb|cleaninstruction|);
		\item Libertar toda a memória (\verb|memdestroy|);
		\item Alocar memória para guardar instruções (\verb|memalloc|);
		\item Libertar a memória associada a um processo (\verb|memfree|);
		\item Obter as instruções a partir de um ficheiro \verb|*.prg|, as quais podem ser alocadas \textit{à posteriori} com o método \verb|memalloc|.
	\end{itemize}
	
	De notar que foi utilizado o \textbf{algoritmo \textit{first-fit}} para alocar memória no método \verb|memalloc|. Portanto, as instruções são armazenadas no primeiro bloco disponível com o tamanho necessário.
	
	
	\chapter{Gestão da memória dinâmica}
	\label{sec:heap}
	
	Para a segunda parte do projeto foi adicionado um segundo tipo de memória: uma memória dinâmica, a par da \textit{heap memory} comummente existente nos sistemas operativos modernos, constituída por 128 partições de 2 KB cada.
	
	
	\section{Estruturação}
	\label{ssec:heap:struct}
	
	As estruturas da memória estão definidas em \verb|types.h|, a fim de serem acessíveis em todo o programa:
	\begin{itemize}
		\item \verb|BLOCK|: estrutura que define uma partição (ou um bloco) da memória dinâmica. Um conjunto de partições constituem uma lista ligada.
		\item \verb|HEAP|: estrutura que define uma memória dinâmica, i.e., um conjunto de partições.
	\end{itemize}

	A estrutura \verb|HEAP| contém, portanto, um apontador para a cabeça da lista ligada que representa as partições de memória (\mintinline{c}{BLOCK *blocks}) e um inteiro que indica a capacidade da memória (ou seja, quantas partições contém) (\mintinline{c}{int capacity}). Esta estrutura inclui outros campos, que funcionam como \textit{flags}, que auxiliam na gestão da memória, nomeadamente:
	
	\begin{itemize}
		\item \mintinline{c}{int top}: índice da última partição alocada;
		\item \mintinline{c}{int calls}: número total de chamadas de alocação efetuadas;
		\item \mintinline{c}{int crossed}: número total de partições percorridas pelo algoritmo de alocação;
		\item \mintinline{c}{int negated}: total de chamadas de alocação negadas por falta de espaço ou por número de partições requisitadas foram dos limites permitidos (entre 3 e 10);
		\item \mintinline{c}{float time}: tempo total dispensado nas alocações de memória;
		\item \mintinline{c}{int *pid}: vetor que identifica pelo \ac{PID} a qual processo cada partição está reservada.
	\end{itemize}

	Para representação mais fiel de uma memória física, a gestão dos \ac{PID}s é feita nesta estrutura e não nas partições propriamente ditas. Uma partição não ``sabe'' a que processo pertence, mas sim apenas qual o seu conteúdo em \textit{bits}. Esta tarefa é do gestor de memória, do qual a estrutura \verb|HEAP| é parte integrante.
	
	Por sua ver, a estrutura \verb|BLOCK| representa uma partição. Uma vez que as partições estão numa lista ligada, há apenas dois elementos que esta estrutura inclui:
	
	\begin{itemize}
		\item \mintinline{c}{void *data}: apontador para um bloco de memória de 2 KB de tipo genérico (podendo assim albergar qualquer tipo de conteúdo).
		\item \mintinline{c}{struct heap *next}: apontador para a próxima partição de memória.
	\end{itemize}
	
	Na prática, os recursos das partições (\verb|data|) são alocados mas não utilizados nesta fase do projeto.
	
	A fim de se poder comparar os 4 algoritmos de gestão de memória implementados, foram criadas 4 memórias \textit{heap}, uma por cada algoritmo, conforme mencionado na Secção \ref{sec:main}.
	
	
	\section{Algoritmos de gestão de memória}
	\label{ssec:heap:algoithms}
	
	Os algoritmos\footnote{Não é âmbito do presente relatório explicar cada um dos algoritmos.} implementados foram os seguintes:
	
	\begin{enumerate}
		\item \textit{First-fit};
		\item \textit{Next-fit};
		\item \textit{Best-fit};
		\item \textit{Worst-fit}.
	\end{enumerate}

	Quando um processo residente ou o processo principal do \chronOS~solicitam a alocação de memória \textit{heap}, é feita a alocação nas 4 memórias criadas para cada um dos algoritmos. O método \verb|heapalloc| (residente nos ficheiros \verb|heapmgr.*|) é invocado, recebendo por parâmetros o PID do processo requerente e o número de partições requisitado, sendo este responsável por solicitar a alocação para cada um dos 4 algoritmos e fazer o respetivo \textit{benchmark}. Existe uma função por cada algoritmo que manipula a respetiva memória dinâmica:
	
	\begin{enumerate}
		\item \mintinline{c}{int heapalloc_first(const int pid, const int size);}
		\item \mintinline{c}{int heapalloc_next(const int pid, const int size);}
		\item \mintinline{c}{int heapalloc_best(const int pid, const int size);}
		\item \mintinline{c}{int heapalloc_worst(const int pid, const int size);}
	\end{enumerate}
	
	Todas as 4 funções retornam o número de partições percorridas até encontrar espaço para alocar os recursos, ou, em caso de falha, a constante \verb|HEAP_ALLOC_NOAVAIL| que indica que não há espaço disponível.
	
	A cada algoritmo está definida uma \textit{flag} binária. A função \verb|heapalloc| retorna a soma destas \textit{flags} conforme cada algoritmo teve sucesso ou não.
	
	\begin{minted}[linenos]{c}
#define HEAP_ALG_FIRST 1        // First-fit   0001
#define HEAP_ALG_NEXT  2        // Next-fit    0010
#define HEAP_ALG_BEST  4        // Best-fit    0100
#define HEAP_ALG_WORST 8        // Worst-fit   1000
	\end{minted}
	
	Por exemplo, o código \verb|6| (\verb|0b0110|) indica que a alocação foi feita com sucesso com os algoritmos \textit{next-fit} e \textit{best-fit}, mas não nos restantes. Caso a alocação seja possível com todos os algoritmos, o código \verb|15| (\verb|0b1111|) é retornado. O código é indicado quer em modo \textit{debug} quer em modo \textit{release} numa mensagem do seguinte tipo:
	
	\begin{verbatim}
Allocated 7 block of heap memory (return code = 13)
	\end{verbatim}
	
	Caso seja solicitada a alocação de um número de partições fora do permitido (entre 3 e 10), \verb|heapalloc| retornará \verb|HEAP_ALLOC_OUTOFRANGE| e contará, para fins estatísticos, que foi feita uma chamada de alocação e que esta foi negada por todos os algoritmos.
	
	
	
	\section{Solicitação de alocação pelos processos residentes}
	\label{ssec:heap:alloc}
	
	Um processo residente, carregado em memória previamente a partir de um ficheiro \verb|*.prg|, pode pedir a qualquer momento a alocação de memória \textit{heap}, podendo esta solicitação ser feita mais do que uma vez. Contudo, um pedido de dealocação resulta na dealocação de toda a memória previamente alocada.
	
	Para este fim, foram criadas 2 novas instruções:
	
	\begin{itemize}
		\item \verb|K n|: solicita a alocação de \verb|n| partições de 2 KB;
		\item \verb|F|: solicita a dealocação de toda a memória a si alocada.
	\end{itemize}

	É permitido pedir a dealocação mesmo que não tenha sido feita nenhuma alocação previamente. Tal não gerará qualquer erro.
	
	As letras foram escolhidas com os seguintes critérios:
	
	\begin{itemize}
		\item \verb|K|: provém de \textbf{K}ilobyte (ou \textbf{K}B). Uma vez que a letra \verb|A|, de \textit{\textbf{A}llocation}, já se encontrava reservada, assim como \verb|M|, de \textit{\textbf{m}alloc}, optámos pela letra \verb|K| uma vez que todas as partições têm tamanho fixo.
		
		\item \verb|F|: provém de \textit{\textbf{F}ree}, inspirado pela função com o mesmo nome do C.
	\end{itemize}

	Um exemplo de programa que solicita alocação de 7 partições e dealoca esta memória antes de terminar pode ser o seguinte:
	
	\begin{verbatim}
M 100
K 7
A 10
S 25
F
T
	\end{verbatim}
	
	
	\section{Solicitação de alocação pelo processo principal}
	\label{ssec:heap:request}
	
	O próprio programa \chronOS~pode fazer solicitações de alocação de memória \textit{heap} a fim de testar a fundo a gestão de memória com os 4 algoritmos previamente mencionados. Estas solicitações são decididas com base num algoritmo pseudoaleatório desenvolvido pelo elemento de grupo Igor Nunes (descrito na subsecção \ref{sssec:heap:request:alg}).
	
	Inicialmente \chronOS~faz chamada do método \verb|heaprequest_start|, o qual arranca o gerador de números pseudoaleatórios. É fornecida por argumento uma semente de geração. Esta semente é, por sua vez, fornecida \textbf{opcionalmente} em argumento pela \textit{shell} (conforme sintaxe definida na secção \ref{ssec:main:argv}). Na falta de fornecimento de uma pela linha de comandos, é assumida uma semente com base no tempo (função \verb|time()|).
	
	A cada ciclo de \textit{clock} é feita uma chamada à função \verb|heaprequest|, a qual devolve 1 para solicitar a alocação de memória, ou 0 em caso contrário. Para a solicitação é invocada a função \verb|heapalloc| com o PID do \chronOS (definida em \verb|PID_CHRONOS|), sendo o número de partições definida pela função \verb|heaprequest_size|.
	
	\begin{minted}[linenos]{c}
if (heaprequest()) {
   int size = heaprequest_size();
   int ret = heapalloc(PID_CHRONOS, size);
   // funções de output
}
	\end{minted}
	
	Antes dos relatórios de execução, a memória alocada pelo \chronOS~é devidamente dealocada para evitar \textit{memory leaks} internos (descritos na subsecção \ref{ssec:heap:stat}):
	
	\begin{minted}[linenos]{c}
heapfree(PID_CHRONOS);
	\end{minted}
	
	
	\subsection{Algoritmo pseudoaleatório para solicitação}
	\label{sssec:heap:request:alg}
	
	O algoritmo desenvolvido é inspirado numa simplificação de uma distribuição normal. Um gráfico relativamente semelhante é o definido pela seguinte função (Figura \ref{graph:heap:request:alg}):
	
	\begin{equation}
f(x) = \frac{\sin\left(x - \frac{\pi}{2}\right) + 1}{2}
	\end{equation}
	
	Este gráfico tem uma área sob a curva de \SI{50}{\percent} do retângulo em que se insere, o que é desejado:
	
	\begin{displaymath}
		A_{\textrm{retângulo}} = 1 \times 2\pi = 2\pi
	\end{displaymath}
	
	\begin{displaymath}
		A_{f(x)} = \int_{0}^{2\pi} \frac{\sin\left(x - \frac{\pi}{2}\right) + 1}{2} dx = \pi
	\end{displaymath}
	
	Logo $A_{f(x)} = 0.5 \times A_{\textrm{retângulo}}$.
	
	\begin{figure}[!htbp]
		\centering
		\begin{tikzpicture}
		\begin{axis}[
		xmin=0,xmax=2*pi,xlabel={Ângulo (radianos)},
		xtick={0, 1.5708, 3.14159, 4.7123889, 6.28318},
		xticklabels={$0$, $\frac{\pi}{2}$, $\pi$, $\frac{3\pi}{2}$, $2\pi$},
		ymin=0,ymax=1,ylabel={Probabilidade p}
		]
		\addplot[domain=0:2*pi,samples=200,smooth,thick,blue] {(sin(deg(x-pi/2))+1)/2};
		\end{axis}
		\end{tikzpicture}
		\caption{Gráfico da função $\left(\sin\left(x - \pi / 2\right) + 1\right) / 2$.}
		\label{graph:heap:request:alg}
	\end{figure}
	
	O algoritmo prossegue, então, da seguinte forma:
	
	\begin{enumerate}
		\item Obter um número $x$ pseudoaleatório no intervalo $[0, 2\pi]$;
		\item Calcular a probabilidade $p$ tal que $p = f(x) \times \SI{100}{\percent}$;
		\item Obter um número $r$ pseudoaleatório no intervalo $[0, 100]$;
		\item Caso $0 \leq r \leq p$, devolver 1. Se não, devolver 0.
	\end{enumerate}

	Isto traduz-se, portanto, no seguinte código em linguagem C:

	\begin{minted}[linenos]{c}
#define M_PI acos(-1.0)
int heaprequest(void) {
   double x = (rand() % 6284) / 1000.;
   int p = (int) ((sin(x - M_PI / 2.) + 1.) / 2. * 100.);
   int r = rand() % 101;
   return (p - r >= 0);
}
	\end{minted}
	
	Este algoritmo introduz, desta forma, 2 pontos de pseudo-aleatoriedade.
	
	
	\section{Relatório e estatísticas}
	\label{ssec:heap:stat}
	
	No final da execução do \chronOS~é impressa a estatística de utilização das 4 memórias \textit{heap} (Figura \ref{fig:heapreport}). Esta inclui as seguintes informações:
	
	\begin{itemize}
		\item \verb|Calls|: número total de solicitações de alocação;
		\item \verb|Crossed|: total de partições percorridas durante os processos de alocação;
		\item \verb|Leaks (KB)|: quantidade em KB de memória que não foram libertados pelos respetivos processos (\textit{memory leaks});
		\item \verb|# fragments|: quantidade de fragmentos externos com 1 ou 2 partições de tamanho;
		\item \verb|Alloc avg time|: tempo médio de alocação;
		\item \verb|Perc no-alloc|: percentagem de solicitações de alocação negadas.
	\end{itemize}
	
	Infelizmente, por motivos que não conseguimos até ao momento compreender, o programa contabiliza sempre que demorou 0 \textit{ticks} de \textit{clock} do programa para alocar a memória dinâmica, apesar de esta ser de facto alocada corretamente por cada um dos algoritmos. Testámos diferentes métodos sem sucesso. O que permaneceu no código final envolve a utilização do método \verb|clock()|. Várias referências na Internet argumentam fortemente contra o uso de alternativas como o \verb|gettimeofday()| para este fim, pelo que não os aplicámos.
	
	\begin{figure}[!htbp]
		\centering
		\includegraphics[width=\textwidth]{heapreport}
		\caption{Exemplo de um relatório das memórias dinâmicas.}
		\label{fig:heapreport}
	\end{figure}

	Para o modo de \textit{debug} é feita a impressão de um \textit{dump} das 4 memórias (exemplo na Figura \ref{fig:heapdump}). Tal permite verificar quais os processos que não libertaram os respetivos recursos alocados. Esta informação não é impressa em modo \textit{release} uma vez que pode ser bastante extensa e densa.
	
	\begin{figure}[!htbp]
		\centering
		\includegraphics[scale=0.6]{heapdump}
		\caption{Exemplo de um \textit{dump} de uma das memórias dinâmicas em modo \textit{debug}.}
		\label{fig:heapdump}
	\end{figure}

	
	\chapter{Leitura e execução de um programa}
	\label{sec:program}
	
	No \chronOS, um programa é um conjunto de \textbf{instruções} constituídas por 2 elementos: instrução, e valor inteiro ou nome de processo filho. A estrutura \verb|instruction| definida em \verb|types.h| define, portanto, uma instrução.
	
	O módulo \texttt{instructions} (definido com os respetivos \textit{header file} e ficheiro \verb|*.c|) contém as funções que executam as instruções propriamente ditas, a saber:
	
	\begin{itemize}
		\item \verb|M n|: mudar o valor da variável para $n$;
		\item \verb|A n|: Adicionar $n$ à variável;
		\item \verb|S n|: Subtrair $n$ à variável;
		\item \verb|B|: Bloquear o processo;
		\item \verb|T|: Terminar o processo;
		\item \verb|C n|: Fazer \textit{fork} do processo: o processo pai executa $n$ linhas após a instrução, enquanto o processo filho executa a linha imediatamente a seguir.
		\item \verb|L filename|: Limpar o programa e substituir pelo programa \verb|filename|.
		\item \verb|K n|: Alocar $n$ partições de memória dinâmica ($3 \leq n \leq 10$);
		\item \verb|F|: Dealocar memória dinâmica previamente alocada pelo processo.
	\end{itemize}

	\begin{note}
		As 2 últimas instruções foram definidas pelo grupo e abordadas na secção \ref{ssec:heap:alloc}. As restantes são conforme o definido no enunciado do projeto.
	\end{note}
	
	O método \verb|program_read_from_file| lê as instruções de um programa, guardado num ficheiro com a extensão \verb|*.prg|, e aloca em memória \textit{heap} um vetor de instruções, cujo apontador é retornado como resultado.
	
	O método \verb|memalloc| recebe um vetor deste tipo e a sua dimensão, e assim aloca no bloco de memória \verb|memory| o espaço necessário para armazenar as instruções, fazendo uma cópia.
	
	O vetor alocado por \verb|program_read_from_file| não é libertado pelo método \verb|memalloc|, pelo que um \verb|free| deve ser executado após este último método.
	
	Os algoritmos de escalonamento são responsáveis por invocar o método \verb|run|, o qual executa a próxima instrução de um processo.
	
	
	\chapter{Discussão e conclusão}
	\label{sec:con_futwork}
	
	O projeto \chronOS~permitiu consolidar uma vasta gama de conhecimentos, não só de Sistemas Operativos como também de Estruturas de Dados e da linguagem C \textit{per se}.
	
	A forte estruturação do projeto em diferentes módulos, constituindo o que chamámos de \textit{Runtime Libraries Package} (RTL), permite que este seja facilmente escalável para mais algoritmos de escalonamento, além de ter facilitado a implementação de certas componentes do \textit{core} do programa.
	
	
	\section{Análise SWOT}
	
	Para discussão do projeto \chronOS, propomos a \textbf{análise SWOT} infra-apresentada.
	
	Os pontos fracos e respetivas oportunidades representam pontos potenciais para trabalho futuro.
	
	\subsection{Pontos fortes}
	
	\begin{enumerate}
		\item Código modular e facilmente escalável.
		\item Forte documentação do código.
		\item Criação de dois modos de compilação (\textit{debug} e \textit{release}).
		\item Controlo via ficheiro, manual ou automático.
		\item Leitura de prioridades dos processos.
		\item Implementação dos algoritmos \ac{FCFS}, \ac{SJF}, \textit{round-robin} e \ac{PSA}.
		\item Forte gestão de memória estática com recurso ao algoritmo \textit{first-fit}.
		\item Clara separação entre gestão de processos e gestão de memória.
		\item Implementação de uma \textit{queue} própria para gestão do plano de execução.
		\item Códigos compactos e objetivos para os processos de \textit{fork} e limpeza de processos.
		\item Utilização correta e consistente de listas ligadas para a construção das memórias dinâmicas.
		\item Algoritmo próprio de pseudo-aleatoriedade para solicitações de alocação com fundamento matemático.
		\item Correta alocação e dealocação de memória com recurso aos 4 algoritmos de gestão.
		\item Inexistência de \textit{memory leaks} na execução do \chronOS.
	\end{enumerate}
	
	
	\subsection{Pontos fracos}
	
	\begin{enumerate}
		\item Pouca variedade de algoritmos de escalonamento.
		\item Pouco proveito dos argumentos passados pela \textit{shell} para controlo dos parâmetros do \chronOS.
		\item Inexistência de um algoritmo de escalonamento que tire partido das prioridades de processos lidas a partir do ficheiro \verb|plan.txt|.
	\end{enumerate}
	
	
	\subsection{Oportunidades}
	
	\begin{enumerate}
		\item Implementação de novos algoritmos de escalonamento, tal como \ac{EDF}.
		\item Maior controlo dos parâmetros internos do \chronOS~através de argumentos passados pela \textit{shell}.
		\item Implementação de um algoritmo de escalonamento por prioridades para escalonamento dinâmico.
	\end{enumerate}
	
	
	\subsection{Ameaças}
	
	\begin{enumerate}
		\item Maior dificuldade de leitura do código devido à vasta documentação interna.
		\item Potencial ineficiência de utilização da memória estática devido às desvantagens do \textit{first-fit}.
		\item Utilização de código próprio para \textit{queues} e listas ligadas pode acarretar \textit{bugs} inesperados por não ser um código público devidamente escrutinado e de eficácia comprovada.
		\item Utilização de algoritmos próprios, não verificado por pares, acarreta desvantagens inerentes.
	\end{enumerate}
	
	
	\section{Conclusão}
	
	Os objetivos iniciais propostos foram cumpridos na sua totalidade com o projeto \chronOS. A gestão de processos e a gestão das memórias são operacionais e executam eficazmente as funções elementares.
	
	% Apesar de uma pequena seleção de objetivos não ter sido plenamente alcançada, outros objetivos não inicialmente planeados foram alcançados em seu lugar.
	A forte modulação do código revelou-se uma mais-valia na construção da aplicação \chronOS~e, em particular, na hora do respetivo \textit{debugging}. De igual forma, foi possível construir vários elementos fundamentais com códigos relativamente simples e compactos devido à criação de funções altamente especializadas e independentes.
	
	A forte separação dos gestores de processos e de memória permitiram que não existissem erros e \textit{bugs} cruzados. Ou seja, um \textit{bug} no gestor de processos, por exemplo, não afeta a gestão das memórias.
	
	A expansão deste programa na sua versão atual para albergar novos algoritmos é, portanto, facilmente alcançável no que diz respeito à comunicação entre os vários componentes que constituem o \textit{core} do programa.
	
	Alcançar um código consistente e com a devida separação entre componentes foi uma consequência natural da nossa decisão em modular os referidos componentes do \textit{core} do \chronOS.
	
	Este é, pois, um projeto de elevado interesse para se dar a sua continuidade no futuro para aprofundamento de conhecimentos.
	
	
	\chapter*{Acrónimos}
	\addcontentsline{toc}{chapter}{Acrónimos}
    \label{sec:acron}
    
    \begin{acronym}[FCFS]
    	\acro{CPU}{\emph{Central Processing Unit}}
    	\acro{EDF}{\emph{Earliest deadline first}}
        \acro{FCFS}{\emph{First Come, First Serve}}
        \acro{gcc}{\emph{GNU C Compiler}}
        \acro{PID}{\emph{Process Identifier}}
        \acro{PCB}{\emph{Process Control Block}}
        \acro{PSA}{\emph{Priority Scheduling Algorithm}}
        \acro{SJF}{\emph{Shortest Job First}}
    \end{acronym}
	
\end{document}
