\documentclass[a4paper,11pt,onecolumn,oneside]{article}

\usepackage[english,portuges]{babel}
%\usepackage[utf8]{inputenc}
\usepackage[T1]{fontenc}
\usepackage{siunitx}
\usepackage{graphicx}
%\usepackage{minted}
%\usepackage{amsmath}
%\usepackage{systeme}
\usepackage{hyperref}
\usepackage[printonlyused]{acronym}

\graphicspath{{img/}}

\newcommand{\chronOS}{\textsf{chronOS}}
\newcommand{\git}{\textsf{git}}

\title{
	Sistemas Operativos --- Projeto\\
	\chronOS~--- \textbf{Relatório preliminar}
}

\author{
	\begin{tabular}[!h]{l l}
		39489 & Jorge Miguel Louro Pissarra\\
		41266 & Diogo Castanheira Simões\\
		41381 & Igor Cordeiro Bordalo Nunes
	\end{tabular}
}

\date{\today}

\begin{document}
	\maketitle
	
	\begin{abstract}
		\chronOS~é um programa de simulação de escalonamento de processos e gestão de memória.
		
		Encontra-se implementado o algoritmo FCFS para o escalonamento, assim como o algoritmo \textit{first-fit} para a gestão de memória.
		
		Um plano de execução é lido no arranque do programa e o simulador de CPU, definido para \SI{2}{\hertz}, invoca o método que implementa o algoritmo de escalonamento para decidir a próxima ação, a qual pode ser um \textit{switch} no estado de um processo ou a execução de uma instrução.
		
		Sempre que possível é utilizada a memória \textit{heap} do computador para albergar os componentes do programa. Tal inclui o bloco de memória e a tabela PCB.
		
		Trabalho futuro é necessário para aumentar significativamente as capacidades do \chronOS, incluindo mais algoritmos de escalonamento.
	\end{abstract}
	
	\newpage
	\tableofcontents
	\vspace{1cm}
	\listoffigures
	\newpage
	
	\section{Preâmbulo}
	\label{sec:intro}
	
	Conhecer o coração dos sistemas operativos modernos implica conhecer diversos métodos de gestão de memória e de processos. \chronOS~propõe-se enquanto projeto de simulação de escalonamento a fim de consolidar conhecimentos na área no âmbito da unidade curricular de Sistemas Operativos.
	
	
	\section{Pesquisa e desenvolvimento}
	\label{sec:dev}
	
	\subsection{Linguagem e pesquisa}
	\label{ssec:dev:sota}
	
	Para este projeto foram consideradas as linguagens C, C++ e OCaml. Optou-se pela linguagem C devido à facilidade em se trabalhar a um mais baixo nível e à grande versatilidade que os apontadores permitem. C++ teria sido uma linguagem ideal para usar orientação a objetos, mas a pouca familiaridade com esta forçou-nos a optar pelo C.
	
	Para reforçar a nossa aprendizagem, decidimos implementar tudo \textit{in-house}, \textbf{não} recorrendo a códigos já existentes ou projetos semelhantes na Internet e em repositórios \git~públicos.
	
	
	\subsection{Repositório no GitLab}
	\label{ssec:dev:gitlab}
	
	O projeto encontra-se alojado num servidor GitLab privado no seguinte \textit{link}: \url{https://gitlab.pcdev.pt/inunes/chronos/}
	
	O repositório \git~em causa inclui vários \textit{branches} nos quais nos encontramos a trabalhar, a saber:
	
	\begin{itemize}
		\item \texttt{master}: \textit{branch} principal para onde é feito \textit{merge} de versões finais funcionais;
		\item \texttt{dev}: \textit{branch} de desevolvimento dedicado do elemento de grupo Jorge Pissarra;
		\item \texttt{dev-in}: \textit{branch} de desevolvimento dedicado do elemento de grupo Igor Nunes (regra geral é neste \textit{branch} que se encontram as \textbf{versões compiláveis sem erros com recurso ao \texttt{Makefile}});
		\item \texttt{dev-ds}: \textit{branch} de desevolvimento dedicado do elemento de grupo Diogo Simões;
	\end{itemize}
	

	\subsection{Resumo do estado atual do projeto}
	\label{ssec:dev:summary}
	
	À data de escrita do presente relatório, o programa \chronOS~é capaz de fazer as seguintes tarefas:
	
	\begin{enumerate}
		\item Alocar células de memória e a tabela \texttt{\ac{PCB}};
		\item Ler o ficheiro \texttt{plan.txt} e guardar em memória o plano de execução numa \textit{queue};
		\item Temporizar o sistema de 500 em 500 milissegundos;
		\item Extrair de um ficheiro \texttt{*.prg} as suas instruções;
		\item Alocar em memória as instruções referentes a um programa recorrendo ao \textbf{algoritmo \textit{first-fit}};
		\item Ler um programa para memória e alocar os dados na tabela \ac{PCB} ao tempo exato indicado pelo plano de execução;
		\item Executar as instruções em memória, incluindo o \textit{fork} (instrução \verb|C n|) e o \textit{clean} (instrução \verb|L filename|);
		\item Gerir os processos com recurso ao \textbf{algoritmo \ac{FCFS}};
		\item Imprimir um relatório do estado dos processos;
		\item Imprimir um relatório do estado da memória em modo \textit{debug};
		\item Libertar os recursos associados à memória, à tabela \ac{PCB} e à \textit{queue} do plano de execução.
	\end{enumerate}
	
	Todas as tarefas foram testadas em modo \textit{debug} recorrendo a \textit{sanitizer flags} (descrito na secção \ref{ssec:dev:struct_makefile}) que nos permitem verificar que não existe nenhum \textit{memory leak} no final da execução do programa.
	
	
	\subsection{Estrutura do código e \texttt{Makefile}}
	\label{ssec:dev:struct_makefile}
	
	A pasta de desenvolvimento está estruturada com as seguintes pastas:
	
	\begin{itemize}
		\item \verb|doc|: pasta onde é escrita a documentação, incluindo este mesmo documento em \LaTeX;
		\item \verb|include|: inclui todos os \textit{header files} (ficheiros \verb|*.h|) necessários ao programa;
		\item \verb|src|: inclui todos os ficheiros \verb|*.c| com a implementação dos protótipos declarados nos \textit{header files}.
	\end{itemize}
	
	Por sua vez, o \texttt{Makefile} está construído de forma a possibilitar 2 formas de compilação:
	
	\begin{itemize}
		\item \textbf{\textit{Debug}} (\verb|make debug|): versão de desenvolvimento com mensagens de \textit{debugging} bastante detalhadas;
		
		\item \textbf{\textit{Release}} (\verb|make release|): versão final sem mensagens de \textit{debug} durante a execução do \chronOS.
	\end{itemize}

	O \texttt{Makefile} inclui uma série de \textit{flags} adicionais face ao comum a fim de controlar de forma mais apertada o comportamento do \ac{GCC}, o qual é bastante permissivo por natureza. O modo \textit{debug} inclui \textit{sanitizer flags} que permitem detetar \textit{memory leaks} e detalhar informações acerca de falhas graves na execução relacionadas com acessos indevidos à memória (os clássicos \textit{segmentation faults}).
	
	
	\section{Execução do \chronOS}
	\label{sec:main}
	
	O programa começa por alocar todos os recursos necessários à execução do programa, nomeadamente:
	
	\begin{itemize}
		\item \verb|memory|: bloco de memória, descrito na secção \ref{sec:memory}, constituída por 1000 células;
		\item \verb|pcb|: tabela \ac{PCB}, descrita na secção \ref{sec:process}, constituída por 100 linhas;
		\item \verb|plan|: \textit{queue} com o plano de entrada de processos no sistema, dinamicamente alocada e ajustada.
	\end{itemize}

	O plano de entrada de processos é lido a partir do ficheiro \texttt{plan.txt}, o qual deve estar obrigatoriamente presente. O plano é alocado em \textit{heap memory} numa \textit{queue} do tipo \verb|plan_q|, definida em \verb|plan.h|. É esperado que o ficheiro tenha os dados ordenados por ordem cronológica de entrada. O comportamento do \chronOS~ para um plano não ordenado não está definido.
	
	\chronOS~executa com base num ciclo que simula uma \ac{CPU} com um \textit{clock} de \SI{2}{\hertz}. A cada ciclo de \textit{clock} é considerado que passou 1 unidade de tempo de \ac{CPU}, armazenada em \verb|cputime|, e que permite determinar os tempos aos quais os processos entram em execução, quanto tempo levam a ser executados e em que tempo são terminados.
	
	A cada ciclo de \textit{clock}, um algoritmo de escalonamento é executado, o qual irá definir a próxima ação a ser tomada. Os algoritmos de escalonamento têm o controlo \textit{de facto} da execução das instruções e dos processos, não sendo responsabilidade da função \verb|main| onde se encontra o simulador da \ac{CPU}.
	
	\begin{figure}[!btp]
		\centering
		\includegraphics[width=\textwidth]{pcbreport}
		\label{fig:pcbreport}
		\caption{Exemplo de um relatório da tabela de processos.}
	\end{figure}
	
	\begin{figure}[!btp]
		\centering
		\includegraphics[scale=0.5]{memreport}
		\label{fig:memreport}
		\caption{Exemplo de um relatório da memória em modo \textit{debug}.}
	\end{figure}

	Para terminar, todos os recursos são libertados e é impresso um relatório da tabela \ac{PCB} (Figura \ref{fig:pcbreport}). Em modo de \textit{debug}, e exclusivamente para efeitos de \textit{debugging} no processo de desenvolvimento, é impresso um relatório adicional da memória (Figura \ref{fig:memreport}).
	
	
	
	
	\section{Gestão de processos}
	\label{sec:process}
	
	Um \textbf{processo} inclui as seguintes informações, definidas na estrutura \verb|process| no \textit{header file} \texttt{types.h}:
	
	\begin{itemize}
		\item \verb|name|: nome do processo;
		\item \verb|pid|: \ac{PID} do processo;
		\item \verb|ppid|: \ac{PID} do processo pai;
		\item \verb|context|: variável inteira associada ao processo;
		\item \verb|start|: endereço de memória da 1ª instrução do processo;
		\item \verb|counter|: PC [\textit{Program Counter}] do processo (endereço absoluto no bloco de memória);
		\item \verb|instsize|: número de instruções do processo;
		\item \verb|state|: estado atual do processo;
		\item \verb|priority|: nível de prioridade do processo;
		\item \verb|timelimit|: \textit{burst time} do processo;
		\item \verb|timeinit|: tempo de \ac{CPU} ao qual o processo iniciou (passou ao estado \texttt{STATUS\_READY});
		\item \verb|timeend|: tempo de \ac{CPU} ao qual o processo terminou (passou ao estado \texttt{STATUS\_TERMINATED});
		\item \verb|timeused|: tempo total consumido a ser processado (durante o estado \texttt{STATUS\_RUNNING}).
	\end{itemize}
	
	Neste momento, \chronOS~implementa o modelo de 5 estados de processos, controlado por um método \verb|switchState|. Este método será utilizado por todos os algoritmos de gestão e escalonamento que serão implementados.
	
	Da mesma forma que a gestão de memória, falada na secção \ref{sec:memory}, a gestão de processos está dividida em três componentes:
	
	\begin{itemize} %Create an acronym session to store this
	    \item O \texttt{\ac{PCB}}, que representa a estrutura onde estão armazenados todos os processos;
	    \item As funções definidas no ficheiro \texttt{\ac{PCB}}, que servem para manipular o \texttt{\ac{PCB}};
	    \item Os algoritmos de escalonamento.
	\end{itemize}  
	
	É, portanto, possível efetuar as atuais operações no \ac{PCB}:
	\begin{itemize}
	    \item Criar uma tabela \ac{PCB};
	    % \item Obter o \ac{PID} mais elevado presente;
	    \item Alocar um processo e armazená-lo na tabela \ac{PCB};
	    \item Libertar toda a estrutura;
	    \item Obter a informação de um processo na estrutura dado o seu \ac{PID}.
	\end{itemize}
	
	O único algoritmo de escalonamento implementado na atual \textit{build} do program é o \ac{FCFS}.
	
	
	\subsection{Algoritmo \ac{FCFS}}
	\label{ssec:process:fcfs}
	
	O \textbf{algoritmo \ac{FCFS}} admite os novos processos (estado \texttt{STATUS\_NEW}) na fila de processos prontos (estado \texttt{STATUS\_READY}) e, uma vez no estado pronto, faz o seu \textit{dispatch} para a fila de processos em execução (estado \texttt{STATUS\_RUNNING}). Uma vez terminada a sua execução, é feito o \textit{release} e o processo fica na fila de processos terminados (estado \texttt{STATUS\_TERMINATED}).
	
	Tudo é feito segundo o princípio FIFO [\textit{First In, First Out}], i.e., uma \textit{queue}. Dada a forma como a estrutura \ac{PCB} está definida no nosso projeto, o método \verb|fcfs| necessita apenas de saber o índice atual da tabela \ac{PCB} a fim de saber as próximas ações a tomar segundo o modelo de 5 estados.
	
	
	\section{Gestão de memória}
	\label{sec:memory}
	
	A memória no \chronOS~é, na sua essência, um bloco monolítico constituído células de memória. Cada célula é uma instrução. A estrutura \verb|MEMORY| definida em \texttt{types.h} define este bloco.
	
	A gestão de memória está dividida em duas componentes:
	\begin{itemize}
		\item A \verb|memory|, que representa a estrutura de memória onde são armazenadas as instruções;
		\item As funções definidas no ficheiro \verb|memmgr.c|, que servem para manipular a \verb|memory|.
	\end{itemize}
	
	Na versão do projeto aqui apresentada, é possível efetuar as seguintes operações sobre a \verb|memory|:
	\begin{itemize}
		\item Criar um bloco de memória (\verb|memcreate|);
		\item Limpar instruções na memória (\verb|cleaninstruction|);
		\item Libertar toda a memória (\verb|memdestroy|);
		\item Alocar memória para guardar instruções (\verb|memalloc|);
		\item Libertar a memória associada a um processo (\verb|memfree|);
		\item Obter as instruções a partir de um ficheiro \verb|*.prg|, as quais podem ser alocadas \textit{à posteriori} com o método \verb|memalloc|.
	\end{itemize}
	
	De notar que foi utilizado o \textbf{algoritmo \textit{first-fit}} para alocar memória no método \verb|memalloc|. Portanto, as instruções são armazenadas no primeiro bloco disponível com o tamanho necessário.
	
	
	\section{Leitura e execução de um programa}
	\label{sec:program}
	
	No \chronOS, um programa é um conjunto de \textbf{instruções} constituídas por 3 elementos: instrução, valor inteiro e nome de processo filho. A estrutura \verb|instruction| definida em \verb|types.h| define, portanto, uma instrução.
	
	O módulo \texttt{instructions} (definido com os respetivos \textit{header file} e ficheiro \verb|*.c|) contém as funções que executam as instruções propriamente ditas, conforme definidas em enunciado.
	
	O método \verb|program_read_from_file| lê as instruções de um programa, guardado num ficheiro com a extensão \verb|*.prg|, e aloca em memória \textit{heap} um vetor de instruções, cujo apontador é retornado como resultado.
	
	O método \verb|memalloc| recebe um vetor deste tipo e a sua dimensão, e assim aloca no bloco de memória \verb|memory| o espaço necessário para armazenar as instruções, fazendo uma cópia.
	
	O vetor alocado por \verb|program_read_from_file| não é libertado pelo método \verb|memalloc|, pelo que um \verb|free| deve ser executado após este último método.
	
	Os algoritmos de escalonamento são responsáveis por invocar o método \verb|run|, o qual executa a próxima instrução de um processo.
	
	
	\section{Conclusão e trabalho futuro}
	\label{sec:con_futwork}
	
	Até este ponto do projeto, \chronOS~permitiu consolidar uma vasta gama de conhecimentos, não só de Sistemas Operativos como também de Estruturas de Dados e da linguagem C \textit{per se}.
	
	A forte estruturação do projeto em diferentes módulos, constituindo o que chamámos de \textit{Runtime Libraries Package} (RTL), permite que este seja facilmente escalável para mais algoritmos de escalonamento, além de ter facilitado a implementação de certas componentes do \textit{core} do programa.
	
	\vspace{0.6cm}
	
	Para trabalho futuro, propõem-se os seguintes pontos:
	
	\begin{itemize}
		\item Implementação efetiva do controlo do \chronOS~a partir de um ficheiro \texttt{contro.txt} (ou do \texttt{stdin} para \textit{debuggin});
		\item Melhorias no algoritmo \ac{FCFS} para permitir escalonamento a longo prazo;
		\item Implementação de mais algoritmos de escalonamento (e.g. \ac{SJF});
		\item Leitura opcional de prioridades de processos a partir do ficheiro \texttt{plan.txt};
		\item Adição de maior controlo sobre os parâmetros do \chronOS~através de argumentos passados ao programa, tais como:
		\begin{itemize}
			\item \textit{Clock} do simulador da \ac{CPU};
			\item Algoritmo de escalonamento a ser executado;
			\item Tamanho do bloco de memória;
			\item Tamanho da tabela \ac{PCB}.
		\end{itemize}
	
		\item Desfragmentador de memória (método \verb|memdefrag|).
	\end{itemize}
	
	\vspace{1cm}
	
	\section*{Acrónimos}
	\addcontentsline{toc}{section}{Acrónimos}
    \label{sec:acron}
    
    \begin{acronym}[FCFS]
    	\acro{CPU}{\emph{Central Processing Unit}}
        \acro{FCFS}{\emph{First Come, First Serve}}
        \acro{GCC}{\emph{GNU C Compiler}}
        \acro{PID}{\emph{Process Identifier}}
        \acro{PCB}{\emph{Process Control Block}}
        \acro{SJF}{\emph{Shortest Job First}}
    \end{acronym}
	
\end{document}
