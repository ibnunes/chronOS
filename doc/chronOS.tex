\documentclass[a4paper,11pt,onecolumn,oneside]{article}

\usepackage[english,portuges]{babel}
%\usepackage[utf8]{inputenc}
\usepackage[T1]{fontenc}
%\usepackage{siunitx}
\usepackage{graphicx}
\usepackage{minted}
%\usepackage{amsmath}
%\usepackage{systeme}
\usepackage{hyperref}

\newcommand{\chronOS}{\textsf{chronOS}}
\newcommand{\git}{\textsf{git}}

\title{
	Sistemas Operativos --- Projeto\\
	\chronOS~--- \textbf{Relatório preliminar}
}

\author{
	\begin{tabular}[!h]{l l}
		39489 & Jorge Miguel Louro Pissarra\\
		41266 & Diogo Castanheira Simões\\
		41381 & Igor Cordeiro Bordalo Nunes
	\end{tabular}
}

\date{\today}

\begin{document}
	\maketitle
	
	\begin{abstract}
		\chronOS~é um programa de simulação de \textit{scheduling} de processos e gestão de memória.
		
		O presente documento é um relatório preliminar do estado do projeto à data de escrita deste.
		
		(TODO)
	\end{abstract}
	
	%\tableofcontents
	%\listoffigures
	%\newpage
	
	\section{Repositório no GitLab}
	
	O projeto encontra-se alojado num servidor GitLab privado no seguinte \textit{link}: \url{https://gitlab.pcdev.pt/inunes/chronos/}
	
	O repositório \git~em causa inclui vários \textit{branches} nos quais nos encontramos a trabalhar, a saber:
	
	\begin{itemize}
		\item \texttt{master}: \textit{branch} principal para onde é feito \textit{merge} de versões finais funcionais;
		\item \texttt{dev}: \textit{branch} de desevolvimento dedicado do elemento de grupo Jorge Pissarra;
		\item \texttt{dev-in}: \textit{branch} de desevolvimento dedicado do elemento de grupo Igor Nunes (regra geral é neste \textit{branch} que se encontram as \textbf{versões compiláveis sem erros com recurso ao \texttt{Makefile}});
		\item \texttt{dev-ds}: \textit{branch} de desevolvimento dedicado do elemento de grupo Diogo Simões;
	\end{itemize}


	\section{Resumo do estado atual do projeto}
	
	(TODO)
	
\end{document}
